\documentclass{article}
\usepackage[utf8]{inputenc}
\usepackage[russian]{babel}
\usepackage{cmap}
\usepackage{pgfplots}

\begin{document}
\large{Отчет по ДЗ №2 Исследование кеша процессора}

\begin{tikzpicture}
\begin{axis} [title = Последовательный доступ, width = 500, grid = major, colormap/greenyellow]
\addplot coordinates {
(7.584963,3.113701)
(8.169925,3.061789)
(8.754888,2.996345)
(9.339850,2.999622)
(9.906891,3.003691)
(10.491853,3.004645)
(11.076816,2.997392)
(11.661778,3.009469)
(12.243174,3.003628)
(12.828136,3.004856)
(13.411511,3.004030)
(13.996473,3.008579)
(14.580730,3.249004)
(15.165221,5.062686)
(15.749869,5.063669)
(16.334832,5.056038)
(16.919794,5.077741)
(17.504757,5.105519)
(18.089657,5.140049)
(18.674578,5.140965)
(19.259541,5.144305)
(19.844485,5.139352)
(20.429448,5.294766)
(21.014402,5.164502)
(21.599364,5.211608)
(22.184323,5.289645)
(22.769286,5.326485)
(23.354248,5.395012)
(23.939210,5.318168)
(24.524172,5.334125)
(25.109134,5.306569)
(25.694096,5.298601)
(26.279058,5.331363)
(26.864021,5.298858)
(27.448983,5.281176)
(28.033946,5.324700)
};
\end{axis}
\end{tikzpicture}

\begin{tikzpicture}
\begin{axis} [title = Случайный доступ, width = 500, grid = major]
\addplot coordinates {
(7.584963,3.170234)
(8.169925,3.194395)
(8.754888,3.240234)
(9.339850,3.195977)
(9.906891,3.200527)
(10.491853,3.174941)
(11.076816,3.206172)
(11.661778,3.187695)
(12.243174,3.238027)
(12.828136,3.216973)
(13.411511,3.314492)
(13.996473,3.469023)
(14.580730,4.762988)
(15.165221,5.144941)
(15.749869,5.359082)
(16.334832,5.890723)
(16.919794,8.158047)
(17.504757,9.417051)
(18.089657,10.244219)
(18.674578,10.950449)
(19.259541,11.784551)
(19.844485,78.506484)
(20.429448,89.584824)
(21.014402,96.073379)
(21.599364,99.368359)
(22.184323,102.865273)
(22.769286,104.320371)
(23.354248,107.522930)
(23.939210,109.390449)
(24.524172,119.896250)
};
\end{axis}
\end{tikzpicture}



\end{document}
